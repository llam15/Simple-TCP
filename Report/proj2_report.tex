\documentclass{article}
\usepackage[margin=1in]{geometry}
\usepackage{enumerate}

\begin{document}

% Don't change this portion-- it will format the section below for you. :)
\newcommand{\coverpage}{
  \onecolumn
  \begin{center}
    \LARGE\labtitle \\ \bigskip \bigskip \large\name
    \\ \bigskip \labdate
    \end{center}
  \setcounter{page}{1} }

%% Change these variables
\newcommand{\labtitle}{CS118: Project 2 Report}
\newcommand{\name}{Leslie Lam, 804-302-387 \\ Dominic Mortel, 904-287-174 \\
Kevin Huynh, 704-283-105}
\newcommand{\labdate}{June 5, 2016}

%% This creates your lab cover page.
\coverpage

\section{Design of the TCP Header}
    We created a class to implement the TCP header described in the spec. The
    class contains 4 uint16\_t variables: sequence number, acknowledgement
    number, window, and flags. This class was used to make the packaging and
    unpackaging of TCP headers easier. The two important functions of the class
    are encode and decode. The encode function generates a vector chars, which
    contains the TCP Header information in the correct order (big-endian). This
    was achieved by using some bit-wise operations to extract the first and
    last bytes of each uint16\_t variable and pushing them onto the vector. The
    decode function takes a vector of chars and fills in the variables of the
    TCPHeader class. This is done by examining each pair of chars in the vector
    and combining them into one uint16\_t variable.

\section{Design of the Client}
    The client first creates a UDP socket using the IP address and the port
    number of the server. The socket options are set to add a timeout option of
    500 milliseconds. Using a TCP Header class that we created and a vector
    buffer to hold the object, the client sends a UDP packet consisting of the
    TCP Header with the appropriate flags set for SYN using sendto. Then the
    client opens up a file named “received.data” to eventually receive data
    from the server. The client then goes into a while loop. Using
    gettimeofday, the client repeatedly sends SYN packets to the server once
    every 500 milliseconds until it receives a SYN/ACK. Next, the client uses
    recvfrom to receive packets from the server. The client keeps track of the
    next expected packet using the sequence number of incoming packets. It
    calculates the next expected sequence number using the previous sequence
    number and the size of the payload. The client uses modulo to implement
    wrap-around of the sequence numbers. After decoding the vector buffer to
    obtain the TCP Header and the payload, we use a series of if-else
    statements to categorize the packet using the flags of the TCP Header. If
    the packet is a SYN/ACK packet, the client sends an ACK to start the
    download process. If the packet is a regular ACK packet, the clients
    appends to the downloaded file only if the sequence number of the packet is
    the next expected packet. Thus, if the sequence number is not the next
    expected packet, it is discarded and not buffered. Next, the client sends
    an ACK packet to the server with the ACK number set to the next expected
    packet from the server. If the client receives a FIN packet, it sends a
    FIN/ACK packet and prepares to terminate the connection. Using
    gettimeofday, if the client does not receive a final ACK after sending the
    FIN/ACK after 500 milliseconds, it resends another FI N/ACK. Lastly, if the
    client receives an ACK packet with the correct sequence number, it will
    terminate the connection and close the file.
    
    Some problems that we faced implementing the client was how to implement
    timeout for sending the initial SYN and the final FIN/ACK to start and
    terminate the connection. Another problem that we faced was how to
    correctly calculate the sequence number so that the file is only appended
    to when the sequence number is correct.

\section{Design of the Server}
    The server first creates a UDP socket that listens to the given port number
    and all network addresses of the host. A recv timeout of 500 milliseconds
    is added via socket options, which causes all recv from calls to timeout
    after 500 milliseconds. The server has three states: connection set up
    state, packet transfer state, and closing connection state.

    In the first state, the server waits for a SYN packet. After receiving a
    packet with the SYN flag set, the server responds with a SYN-ACK packet.
    When it receives the next ACK, the server will begin transferring the file.

    At this point, the server is in packet transfer state. First, the server
    calculates the window, WND, from the minimum of the congestion window,
    receiver window, and (max sequence number + 1)/2. This window tells the
    server how many packets can be sent at a time. Then the server tries to
    send as many packets as possible, each with a payload size of 1024 bytes
    (for a total packet size of 1032 bytes). The server sends each packet at
    once and then waits for ACKs from the client. A timer is used to keep track
    of the latest received ACK. If the server does not receive an ACK for the
    oldest packet within 500 milliseconds, then timeout occurs. If a correct
    ACK is received, the server's congestion window is either doubled (if in
    slow start state) or increased by MSS(MSS/CWND) (if in congestion avoidance
    state).

    When the server times out, it sets the slow start threshold (SSTHRESH) to
    half of the current congestion window, or 1024 bytes if half the congestion
    window is less than 1024 bytes. The congestion window is then reduced to
    1024 bytes, and the server begins retransmitting from the last unacked
    packet.

    After the server has finished transmitting the whole file, then the server
    then waits to receive all of the remaining ACKs. When all of the packets
    have been ACKed, the server sends a packet with the FIN flag set,
    signalling the beginning of the connection teardown. After sending the FIN,
    the server waits for the client's FIN/ACK response. If no FIN/ACK is
    received before the timeout, the server will retransmit the FIN. Finally,
    after receiving the FIN/ACK, the server will send one final ACK. After
    sending this ACK, the server then waits 3 RTO (1.5 seconds); if the server
    receives any packets during this time, it will check the flags to see if it
    is a FIN/ACK. If the packet is a FIN/ACK, the server will retransmit the
    final ACK. If the server does not receive anything for 3 RTO, then the
    server will close the socket and terminate.

\section{Problems Encountered}
    We encountered several problems during this project. 
    \begin{enumerate}
        \item We had trouble sending a basic message from the server to the
            client at first. However, this was fixed after updating our
            Vagrantfile.
        \item We also struggled with how to implement the timeout
            functionality. At first we thought that the ACKs were not
            cumulative, so we were thinking about ways to set a timer for each
            different packet. We first thought that maybe we could have a queue
            of structs containing information about the packet number and a
            corresponding timeout time. However, this proved very difficult to
            implement. We later found out that the ACKs were cumulative, so we
            followed the approach outlined in the textbook, where only one
            timer is used. When a client receives a packet out of order, then
            it will simply retransmit the same ACK again and drop the erronous
            packet. The server then keeps a timer corresponding to the oldest
            unacked packet. If the timer expires before an ACK is received for
            that packet, then the server will go into timeout and retransmit
            that packet. If a correct ACK is received, then the server will
            restart the timer if there are any unacked packets left.
        \item We had an issue with our transferred file being corrupted when
        	the server had to retransmit packets. At first the issue was an
        	off-by-one error; we had a variable that kept track of the location
        	in the file of the oldest unacked packet. If the server timed out,
        	then it would restart transmissions at that locatioin in the file.
        	However, our variable was actually one byte ahead of the correct
        	location, causing all retransmitted packets to be one byte ahead.
        	After we fixed this issue, we realized that we had a problem with
        	our sequence numbers. If a packet is lost, the client will continue
        	to resend the same ACK until it gets a packet with the correct
        	sequence number. However, our server window was being calculated
        	wrong, and the sequence numbers would wrap around, such that the
        	client would receive a packet with the correct sequence number, but
        	not the correct payload. Thus the file would be missing a chunk of
        	bytes in the middle. This was fixed by ensuring that our window
        	including the number of unacked packets was smaller than (max
        	sequence number + 1)/2.
        \item We also had an issue where the server and client would enter an
        	infinite loop when tearing down the connection. This was also due
        	to another off-by-one error in our calculation of sequence numbers
        	and ACKs. After fixing this error, the server and client terminated
        	properly.
    \end{enumerate}

\section{Testing}
    To test our implementation we sent several different types of packets from 
    client to server:
    \begin{itemize}
        \item Basic TCP Header Packet: we sent over basic packet header to test
            that the UDP socket was sending and receiving properly.  This also
            tested to see if our TCPHeader class would properly encode and
            decode byte streams read of the socket
        \item Files smaller than 1 MSS: next we tested sending a packet that
            contained a payload, in this case we used the README.md file in the
            directory. This tested that our code could handle a payload
            concatenated to a header and also that we can properly read in a
            file
        \item Files larger than 1 MSS: now we test files larger than one
            minimum segment size to determine if we can send multiple packets
            from the server to the client. At the same time we print out a
            status message per sent and received packet for both the client and
            the server to ensure that we are observering the proper sequence
            numbers and general behavior, such as sequence numbers wrapping
            around.
    \end{itemize}
    After sending the file, we tested using cout statements to see if the
    server was sending the FIN packet, receiving the FIN-ACK packet, sending
    the final ACK and then closing the socket.  Then once that was done we used
    ``diff" to compare the sent file to the receivied file for corruption.

    The next step is to test the timeout:
    \begin{enumerate}
        \item We created a function that would be called on timeout and
            had it print out a message indicating that a timeout has occured.
        \item After adding in timers that would reset after a succesful sent
            packet we test that no timeouts occur when there is no packet loss
            (timeout from either the recv call or our own timer) and no
            corruption in the file
        \item Next we introduce X\% packet loss and Y ms latency using
            options from the vagrant file to see if the basic timeout and 
            the recv function will report any timeout occurring 
        \item Finally we use our fully implemented timeout function with X\% 
            packet loss and Y ms latency to test that we are observing the
            proper sequence number increase, the proper congestion window size
            increase and decrease on packet received and packet loss, the
            proper retransmission of lost packets (and that they have the
            ``Retransmission") string printed out in the status, and lastly
            that the resulting file on the client side is not corrupted.
    \end{enumerate}

    In general, we mainly used cout statements to determine if we were seeing
    the correct behavior, printing out values and deciding if those are the
    values we needed to see.  For example we had file corruption, but this was
    due to the index we were using to read in the file not being set to the
    correct position on retransmission, and printing out the values helped
    narrow the problem down.

    Overall, we determined that our server and client do transfer files
    properly. However, for a file of 4MB, it may take up to 20 minutes if there
    is significant packet loss. This is because if our SSTHRESH drops down to
    1024, then the server will always be in Congestion Avoidance mode, making
    it difficult to transfer packets quickly. Some of the results of our tests
    are as follows:
    \begin{itemize}
        \item 5\% packet loss: $\sim$2 minutes
        \item 10\% packet loss: $\sim$6 minutes
        \item 15\% packet loss: $\sim$10 minutes
        \item 15\% packet loss, 4ms delay: $\sim$13 minutes
    \end{itemize}

\section{Contributions}
    Leslie Lam (804-302-387) and Dominic Mortel (904-287-174) wrote the web
    server. Kevin Huynh (704-283-105) wrote the web client. We all helped debug
    the client and server. Each wrote the corresponding sections of the report.

\end{document}